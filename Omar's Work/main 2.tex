\documentclass{article}
\usepackage[utf8]{inputenc}
\usepackage[english]{babel}
\usepackage[]{amsthm} %lets us use \begin{proof}
\usepackage[]{amssymb} %gives us the character \varnothing
\usepackage[margin=0.75in]{geometry}
\usepackage{amsmath}
\usepackage{graphicx}
\usepackage{derivative}
\usepackage{booktabs}
\usepackage[dvipsnames]{xcolor}
\usepackage{comment}
\usepackage{fancyhdr}
\usepackage{breqn}
\usepackage{float}
\usepackage[shortlabels]{enumitem}
\graphicspath{ {./images/} }

\title{ECON 6090-Microeconomic Theory. TA Section 6}
\author{Omar Andujar}
\date{\today}
%This information doesn't actually show up on your document unless you use the maketitle command below

\begin{document}
\raggedright
\pagestyle{fancy}
\fancyhf{}
%\fancyhead[R]{Omar Andujar}
%\fancyhead[L]{Homework 5}
%\fancyfoot[L]{\hrule Some of the problems were discussed with }

\maketitle %This command prints the title based on the information entered above

\section*{In Section notes}
\textbf{Cost Minimization Problem (CMP)}\\
\[\text{cost function } c(w,q) = \min_z w \cdot z \text{ subject to } f(z) = q\]
Conditional input demand: $z(w,q)$\\
\bigskip
Properties of the cost function:
\begin{enumerate}
    \item c(w,q) is HoD1 in w.
    \item c(w,q) is nondecreasing in q if we assume free disposal.
    \item c(w,q) is concave in w.
    \item If f(z) is HoD k, then c(w,q) is HoD $\frac{1}{k}$ in $q$.
\end{enumerate}
\textbf{Profit Maximization Problem (PMP)}\\
\[\pi(p,w) = \max_x pf(x) -wx\]
Where $x(p,w)$ is the input demand and $y(p,w)$ is the output supply.\\
\bigskip
Properties:
\begin{enumerate}
    \item Nondecreasing in p and nonincreasing in $w_i$ for all i.
    \item HoD 1 in $(p,w)$.
    \item Convex in $(p,w)$.
    \item Continuous on $\mathbb{R}_{++}^1$ x $\mathbb{R}_{++}^n$
\end{enumerate}
\textbf{Derivatives}
\begin{enumerate}
    \item $\frac{\partial \pi(p,w)}{\partial p} = f(x(p,w)) = y(p,w)$
    \item $\frac{\partial \pi(p,w)}{\partial w_i} = -x_i(p,w)$
    \item $\frac{\partial c(w,q)}{\partial w_i} = z_i(w,q)$
    \item $\frac{\partial c(w,q)}{\partial q} = MC \text{(Marginal Cost)}$
\end{enumerate}
\section*{Exercises}

\subsection*{Solving problems with continuum of inputs}
The problem is,
\[\pi(p,w) = \max_{z(j)} pf(z) - \int_{0}^1 w(j)z(j)dj\]
Given the production function \( f(z) = \int_0^1 z(j)^\alpha \, dj \), the profit maximization problem becomes
\[
\pi(p, w) = \max_{z(j)} \left[ p \int_0^1 z(j)^\alpha \, dj - \int_0^1 w(j) z(j) \, dj \right].
\]
Simplifying,
\[
\pi(p, w) = \max_{z(j)} \int_0^1 \left[ p z(j)^\alpha - w(j) z(j) \right] \, dj.
\]
The problem is separable\footnote{For an optimization problem, separability means that the objective function can be expressed as a sum (or integral) of terms, each depending only on a single variable (or a small subset of variables). This property allows the optimization over all variables to be broken down into independent subproblems that can be solved separately for each variable.\\
The objective function is separable because: The term \( p z(j)^\alpha - w(j) z(j) \) depends only on \( z(j) \) for a fixed \( j \). There is no interaction between \( z(j) \) and \( z(k) \) for \( j \neq k \).
}, so the maximization for each \( z(j) \) can be solved independently,

\[
\max_{z(j)} \left[ p z(j)^\alpha - w(j) z(j) \right].
\]
Taking the derivative with respect to \( z(j) \) and setting it to zero (FOC),

\[
\frac{\partial}{\partial z(j)} \left[ p z(j)^\alpha - w(j) z(j) \right] = 0,
\]

\[
p \alpha z(j)^{\alpha - 1} - w(j) = 0.
\]

Solving for \( z(j) \):

\[
z(j) = \left( \frac{w(j)}{p \alpha} \right)^{\frac{1}{\alpha - 1}} = x(j,p,w).
\]
To confirm a maximum, check the second derivative:

\[
\frac{\partial^2}{\partial z(j)^2} \left[ p z(j)^\alpha - w(j) z(j) \right] = p \alpha (\alpha - 1) z(j)^{\alpha - 2}.
\]

Since \( \alpha \in (0, 1) \), the term \( (\alpha - 1) < 0 \), so the second derivative is negative, confirming a maximum.\\
\bigskip

\textbf{Extra (not required):} The optimal allocation is:

\[
z^*(j) = \left( \frac{w(j)}{p \alpha} \right)^{\frac{1}{\alpha - 1}}.
\]

Substitute \( z^*(j) \) into the profit function:

\[
\pi(p, w) = \int_0^1 \left[ p \left( z^*(j) \right)^\alpha - w(j) z^*(j) \right] \, dj.
\]

Substitute \( z^*(j) \) explicitly. First, compute \( \left( z^*(j) \right)^\alpha \):

\[
\left( z^*(j) \right)^\alpha = \left( \frac{w(j)}{p \alpha} \right)^{\frac{\alpha}{\alpha - 1}}.
\]

And:

\[
w(j) z^*(j) = w(j) \left( \frac{w(j)}{p \alpha} \right)^{\frac{1}{\alpha - 1}}.
\]

Thus:

\[
\pi(p, w) = \int_0^1 \left[ p \left( \frac{w(j)}{p \alpha} \right)^{\frac{\alpha}{\alpha - 1}} - w(j) \left( \frac{w(j)}{p \alpha} \right)^{\frac{1}{\alpha - 1}} \right] \, dj.
\]

Simplify further to compute the explicit profit if \( w(j) \) is specified.

\subsection*{A question from a past Q exam}
\begin{enumerate}[(a)]
    \item The problem is
    \[E_p[\pi(p,w)] = \max_x E_p[px_1^{\alpha}x_2^{\beta}-w_1x_1-w_2x_2]\]
    \[=\max_x E_p[p]x_1^{\alpha}x_2^{\beta}-w_1x_1-w_2x_2\]
    Where $E[p]= \delta p_1+(1-\delta)p_2$.\\
    \bigskip
    The FOCs are,
    \[\text{ $x_1$: } \alpha E(p)x_1^{\alpha-1}x_2^{\beta} = w_1\]
    \[\text{ $x_2$: } \beta E(p)x_1^{\alpha}x_2^{\beta-1} = w_2\]
    Then,
    \[x_1^* = E(p)^{\frac{1}{1-\alpha-\beta}}\alpha^{\frac{1-\beta}{1-\alpha-\beta}}\beta^{\frac{\beta}{1-\alpha-\beta}}w_1^{\frac{\beta-1}{1-\alpha-\beta}}w_2^{\frac{-\beta}{1-\alpha-\beta}}\]
    \[x_2^* = E(p)^{\frac{1}{1-\alpha-\beta}}\alpha^{\frac{\alpha}{1-\alpha-\beta}}\beta^{\frac{1-\alpha}{1-\alpha-\beta}}w_1^{\frac{-\alpha}{1-\alpha-\beta}}w_2^{\frac{\alpha-1}{1-\alpha-\beta}}\]
    And the optimal output is,
    \[q(E(p),w) = f(x^*) = E(p)^{\frac{\alpha+\beta}{1-\alpha-\beta}}\alpha^{\frac{\alpha}{1-\alpha-\beta}}\beta^{\frac{\beta}{1-\alpha-\beta}}w_1^{\frac{-\alpha}{1-\alpha-\beta}}w_2^{\frac{\beta}{1-\alpha-\beta}}\]

    \item 
    In this case we replace $E(p) = p_1$ and $E(p) = p_2$ respectively, and get,
    \[q(p_1,w) = f(x^*) = p_1^{\frac{\alpha+\beta}{1-\alpha-\beta}}\alpha^{\frac{\alpha}{1-\alpha-\beta}}\beta^{\frac{\beta}{1-\alpha-\beta}}w_1^{\frac{-\alpha}{1-\alpha-\beta}}w_2^{\frac{\beta}{1-\alpha-\beta}}\]
    \[q(p_2,w) = f(x^*) = p_2^{\frac{\alpha+\beta}{1-\alpha-\beta}}\alpha^{\frac{\alpha}{1-\alpha-\beta}}\beta^{\frac{\beta}{1-\alpha-\beta}}w_1^{\frac{-\alpha}{1-\alpha-\beta}}w_2^{\frac{\beta}{1-\alpha-\beta}}\]

    \item 
    Let $g(w) = \alpha^{\frac{\alpha}{1-\alpha-\beta}}\beta^{\frac{\beta}{1-\alpha-\beta}}w_1^{\frac{-\alpha}{1-\alpha-\beta}}w_2^{\frac{\beta}{1-\alpha-\beta}}$, and $\alpha + \beta = \frac{1}{2}$, then,
    \[\implies \frac{\alpha+\beta}{1-\alpha-\beta} = 1\]
    \[\implies q(E(p),w) = E(p)\alpha^{\frac{\alpha}{1-\alpha-\beta}}\beta^{\frac{\beta}{1-\alpha-\beta}}w_1^{\frac{-\alpha}{1-\alpha-\beta}}w_2^{\frac{\beta}{1-\alpha-\beta}}\]
    \[=(\delta p_1+(1-\delta)p_2)\alpha^{\frac{\alpha}{1-\alpha-\beta}}\beta^{\frac{\beta}{1-\alpha-\beta}}w_1^{\frac{-\alpha}{1-\alpha-\beta}}w_2^{\frac{\beta}{1-\alpha-\beta}}\]
    \[=\delta q(p_1,w) + (1-\delta)q(p_2,w)\]
    Therefore, the expectation of the outputs in part (b) equals the output in part (a).

    \item 
    By a) we have $E_p[\pi(p,w)] = \pi(\delta p_1+(1-\delta)p_2,w)$.\\
    By expected b) we have $\delta q(p_1,w) + (1-\delta)q(p_2,w)$\\
    Since the profit function is convex in $(p,w)$, it is convex in $p$, and we can apply Jensen's inequality to conclude,
    \[(a) \leq \text{expected } (b)\]
    \end{enumerate}

\subsection*{The Cost Function}
We are given $C(w,q)|_{q=0} = C(w,0)=0$ and Marginal Cost = $\frac{\partial C(w,q)}{\partial q} =k$.\\
Now we take integral back on $q$,
\[\implies C(w,q) = kq + T(w)\]
Given our initial condition $C(w,0) = T(w) = 0$,
\[\implies C(w,q) kq\]
Which means that it is HoD1 in $q$. Therefore, the production function is HoD1 in $(p,w)$.\\
Some examples are:
\begin{enumerate}
    \item Cobb-Douglas
    \[f(x_1,...,x_n) = \Pi_{i=1}^nx_i^{\beta_i} \text{ where } \sum_{i=1}^n \beta_i = 1\]
    \item Leontief
    \[f(x_1,...,x_n) = \min\{\beta_1 x_1,...,\beta_n x_n\}\]
    \item Constant Elasticity of Substitution (CES)
    \[f(x_1,...,x_n) = (\sum_{i=1}^n \beta_i x_i^r)^{\frac{1}{r}}\]
\end{enumerate}


\end{document}

