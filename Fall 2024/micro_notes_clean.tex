\documentclass[12pt]{article}

\input{../notes_preamble.tex}

\title{Microeconomics I Notes}
\author{Gabe Sekeres \& Omar Andujar}
\course{ECON 6090}
\term{Fall 2024}
\begin{document}
\maketitle

\section*{Introduction}

We are creating this set of unified notes for ECON 6090: Microeconomics I, as taught at Cornell University in the Fall 2024 semester. Due to unforeseen departmental circumstances, this course was taught by six different professors (\href{https://easley.economics.cornell.edu/}{David Easley}, \href{https://philippkircher.com/}{Philipp Kircher}, \href{https://adamharris.phd/}{Adam Harris}, \href{https://sites.santafe.edu/~leb/}{Larry Blume}, \href{https://barseghyan.economics.cornell.edu/}{Levon Barseghyan}, and \href{https://www.mbattaglini.com/}{Marco Battaglini}). This structure necessarily created some confusion in notation and material, so these notes function as my attempt to create a universe of the material we learned.

We rely heavily on the notes created from Prof. Easley's course, which were originally compiled by \href{https://julienneves.com/}{Julien Manuel Neves} and subsequently updated by \href{https://ruqing-xu.github.io/}{Ruqing Xu} and \href{https://economics.cornell.edu/patrick-ferguson}{Patrick Ferguson}, as well as the excellent TA Sections curated by \href{https://economics.cornell.edu/yuxuan-ma}{Yuxuan Ma} and \href{https://dyson.cornell.edu/programs/graduate/graduate-student-directory/}{Feiyu Wang}. We additionally rely on notes and slides provided by Prof. Harris, slides provided by Prof. Blume, slides from \href{https://blogs.cornell.edu/odonoghue/}{Ted O'Donoghue} provided by Prof. Barseghyan, and notes provided by Prof. Battaglini. These notes are supplemented with the canonical \href{https://global.oup.com/academic/product/microeconomic-theory-9780195073409?cc=us&lang=en&}{Microeconomic Theory} textbook by \href{https://www.upf.edu/web/andreu-mas-colell}{Andreu Mas-Colell}, \href{https://mitsloan.mit.edu/faculty/directory/michael-whinston}{Michael Whinston}, and \href{https://www.hbs.edu/faculty/Pages/profile.aspx?facId=6466}{Jerry Green} (hereafter, MWG); a classic analysis textbook, \href{https://store.doverpublications.com/products/9780486477664}{Foundations of Mathematical Analysis} by \href{https://condor.depaul.edu/~rjohnson/}{Richard Johnsonbaugh} and \href{https://www.mathgenealogy.org/id.php?id=12494}{W.E. Pfaffenberger}; and the excellent Mathematics notes provided by \href{https://www.takumahabu.com/Economics}{Takuma Habu}. All mistakes are our own.

We occasionally make reference to the Stanford ECON 202 notes, created by \href{https://www.gsb.stanford.edu/faculty-research/faculty/jonathan-levin}{Jonathan Levin}, \href{https://web.stanford.edu/~isegal/}{Ilya Segal}, \href{https://milgrom.people.stanford.edu/}{Paul Milgrom}, and \href{https://sites.google.com/site/ravijagadeesan/}{Ravi Jagadeesan}. This will mainly be if there exists intuition that we believe is helpful.

\paragraph{Notation.} A large part of this project is an attempt to unify the notation used by our separate professors. We default to the notation used in the Easley notes, then to MWG, and then use our own judgement. New definitions will have a word highlighted in \blue{blue}, and certain (named) theorems will be denoted in \red{red}. 

\paragraph{Structure.} The course (and these notes) are organized as follows. Prof. Easley taught an introduction to choice theory, Section~\ref{sec:easley}. Prof. Kircher taught consumer theory, Section~\ref{sec:kircher}. Prof. Harris taught producer theory, and some concepts of market failures, Section~\ref{sec:harris}. Prof. Blume introduced the theory of choice under uncertainty, Section~\ref{sec:blume}, and Prof. Barseghyan continued with theoretical applications for uncertainty and expected utility maximization, Section~\ref{sec:barseghyan}. Prof. Battaglini taught on information theory, Section~\ref{sec:battaglini}. We also include here exercises with solutions, divided into the various sections and sources. This is Section~\ref{sec:exercises}.


\newpage
\section{Choice (Easley)}\label{sec:easley}

\subsection{Preference Theory}

\begin{assumption}
	Let $X$ be a finite set of objects. 
\end{assumption}

\begin{definition}
	Define $\succsim$, a \blue{preference relation} on $X$, as $x \succsim y \Longleftrightarrow x$ is \emph{at least as good as} $y$, for $x,y \in X$. $\succsim$ is a binary relation.
\end{definition}

\begin{definition}
	$x$ is \blue{strictly preferred} to $y$, denoted as $x \succ y$, if $x \succsim y$ and $y \not\succsim x$.
\end{definition}
\begin{definition}
	$x$ is \blue{indifferent} to $y$, denoted as $x \sim y$, if $x \succsim y$ and $y \succsim x$.
\end{definition}

\begin{definition}
	A preference relation $\succsim$ is \blue{complete} if $\forall x,y \in X$, either $x\succsim y$, $y \succsim x$, or both.
\end{definition}
\begin{definition}
	A preference relation $\succsim$ is \blue{transitive} if, $\forall x,y,z \in X$ where $x \succsim y$ and $y \succsim z$, $x \succsim z$.
\end{definition}
\begin{definition}
	A preference relation $\succsim$ is \blue{rational} if it is complete and transitive.
\end{definition}
\begin{remark}
	Prof. Easley takes some issues with this definition. The main issue is that there is an English word `rational' that has absolutely nothing to do with it. Hereafter, always read rational as `complete and transitive'.
\end{remark}

\begin{remark}
	These are all of the abstract concepts in choice theory! From here, we will apply them, and see what we can get. 
\end{remark}

\begin{definition}
	(Informal) Define a \blue{choice structure} $C\opt$ over subsets $B \subseteq X$ as $C\opt(B,\succsim) \coloneqq \{x \in B : x \succsim y \forall y \in B\}$.
\end{definition}
\begin{remark}
	Some direct implications:
	\begin{itemize}
		\item[(i)] If $x \in C\opt(B,\succsim)$ and $y \in C\opt(B,\succsim)$, then $x \sim y$.
		
		\item[(ii)] Suppose that $x \in B$, $x \not\in C\opt(B,\succsim)$, and $C\opt(B,\succsim) \ne \emptyset$. Then there exists $y \in B$ such that $y \succ x$. 
	\end{itemize}
\end{remark}

We will now formalize the above.

\begin{definition}
	Let the \blue{power set} of $X$, denoted $\mathcal{P}(X)$, be the set of all subsets of $X$. Note that since $X$ is finite, $\mathcal{P}(X)$ is finite.
\end{definition}

\begin{definition}
	(Formal) A correspondence $C\opt : \mathcal{P}(X) \rightrightarrows X$ is a \blue{choice correspondence} for some (not necessarily complete; not necessarily transitive) preference relation $\succsim$ if $C\opt(B) \subseteq B$ for all $B \subseteq X$.
\end{definition}
\begin{remark}
	This definition is from the Stanford notes -- I find it more intuitive than defining it the other way, but it requires divorcing the choice structure from the preference relation. Some intuition that's helpful for me: Easley's definition starts with the preference relation and then defines the choice correspondence, while Segal's definition starts with the choice correspondence and then applies it to a preference relation. They will (as we will see below) often be equivalent, but it's a subtle distinction. I will denote an arbitrary choice correspondence by $C\opt(\cdot)$ and one connected with a preference relation $\succsim$ by $C\opt(\cdot, \succsim)$.
\end{remark}

\begin{proposition}\label{prop:rational_choice_nonempty}
	If $\succsim$ is a rational preference relation on $X$, then
	\[
	C\opt : \mathcal{P}(X) \setminus \emptyset \to \mathcal{P}(X) \setminus \emptyset 
	\]
	In words, the associated choice correspondence to a rational preference relation is nonempty for nonempty inputs.
\end{proposition}
\begin{remark}
	The Easley notes define power sets slightly differently. This is unnecessary and (I feel) less intuitive.
\end{remark}
\begin{proof}
	Proof by induction on $n = |B|$. Suppose $|B| = 1$, so $B = \{x\}$ for some $x \in X$. Then by completeness, $x \succsim x$, and $C\opt(B,\succsim) = \{x\} \in \mathcal{P}(X) \setminus \emptyset$. Suppose next that for any $Y$ where $|Y| = n$, $C\opt(Y,\succsim)$ is nonempty. Take some arbitrary $B$, where $|B| = n + 1$. Define $B'\coloneqq B \setminus \{x\}$, and let $x'$ be an element of $C\opt(B',\succsim)$, which is nonempty by the inductive hypothesis. By completeness, either $x \succ x'$, $x' \succ x$, or $x \sim x'$. Case by case, we would have that $C\opt(B,\succsim) \in \curll\{x\},C\opt(B',\succsim),C\opt(B',\succsim) \cup \{x\}\curlr \subseteq \mathcal{P}(X)$, by transitivity.
\end{proof}

\begin{definition}
	$C\opt$ satisfies \blue{Sen's $\alpha$}, also known as \blue{independence of irrelevant alternatives}, if $x \in A \subseteq B$ and $x \in C\opt(B,\succsim)$ implies that $x \in C\opt(A,\succsim)$.
\end{definition}
\begin{remark}
	The classical example of a preference relation that violates Sen's $\alpha$ is `choosing the second-cheapest wine.' It should be fairly clear to see why this violates Sen's $\alpha$. Is it a rational preference relation?
\end{remark}

\begin{proposition}\label{prop:rational_alpha}
	If $\succsim$ is a rational preference relation, then $C\opt(\cdot,\succsim)$ satisfies Sen's $\alpha$.
\end{proposition}
\begin{proof}
	The result is trivially true if $A = B$. Suppose that $A \subset B$. Let $x \in C\opt(B,\succsim)$. Then $x \succsim y$ for all $y \in B$. In particular, if $y \in A \subseteq B$, then $x \succsim y$. Thus, $x \in C\opt(A,\succsim)$.
\end{proof}

\begin{definition}
	$C\opt$ satisfies \blue{Sen's $\beta$}, also known as \blue{expansion consistency}, if $x,y \in C\opt(A,\succsim)$, $A \subseteq B$, and $y \in C\opt(B,\succsim)$ implies that $x \in C\opt(B,\succsim)$.
\end{definition}

\begin{remark}
	I couldn't find a classical example violating Sen's $\beta$, but a simple one is as follows: assume that the waiter offers you French or Italian wine. You are indifferent between them, but then they remember that they also have California wine. You say 	`in that case, I'll have the French wine'. Again, this directly violate's Sen's $\beta$, but is it rational? Why or why not?
\end{remark}

\begin{proposition}\label{prop:rational_beta}
	If $\succsim$ is a rational preference relation, then $C\opt(\cdot,\succsim)$ satisfies Sen's $\beta$.
\end{proposition}
\begin{proof}
	Let $x,y \in C\opt(A,\succsim)$, $A \subseteq B$, and $y \in C\opt(B,\succsim)$. Since $x \in C\opt(A,\succsim)$, we have $x \succsim y$ since $y \in A$. Since $y \in C\opt(B,\succsim)$, we have $y\succsim z$ for all $z \in B$. By transitivity, $x \succsim y$ and $y \succsim z$ implies that $x \succsim z$ for all $z \in B$, so $x\in C\opt(B,\succsim)$. 
\end{proof}

\begin{definition}
	$C\opt$ satisfies \blue{Houthaker's weak axiom of revealed preference} (often called either \blue{HWARP} or \blue{HARP}) if for all $A,B \in \mathcal{P}(X)$ if $x,y \in A \cap B$, $x \in C\opt(A,\succsim)$ and $y \in C\opt(B,\succsim)$, then $x \in C\opt(B,\succsim)$ and $y \in C\opt(A,\succsim)$.
\end{definition}

\begin{proposition}\label{prop:alpha_beta_hwarp}
	$C\opt : \mathcal{P} \rightrightarrows X$ satisfies Sen's $\alpha$ and $\beta$ if and only if it satisfies Houthaker's weak axiom of revealed preference.
\end{proposition}
\begin{proof}
	
	\begin{itemize}
		\item[(i)] ($\alpha + \beta \Longrightarrow $ HWARP) Suppose $x,y \in A \cap B \subseteq \mathcal{P}(X)$, $x \in C\opt(A,\succsim)$, and $y \in C\opt(B,\succsim)$. By Sen's $\alpha$, both $x$ and $y$ are in $C\opt(A \cap B,\succsim)$. Then by Sen's $\beta$, $x \in C\opt(B,\succsim)$ and $y \in C\opt(A,\succsim)$.
		
		\item[(ii)] (HWARP $\Longrightarrow \beta$) Say $x,y \in C\opt(A,\succsim)$, $A \subseteq B$ and $y \in C\opt(B,\succsim)$. Because $A = A \cap B$, $x,y \in C\opt(A \cap B,\succsim)$. Applying HWARP, we have that $x \in C\opt(B,\succsim)$.
		
		\item[(iii)] (HWARP $\Longrightarrow \alpha$) Say $x \in A \subseteq B$ and $x \in C\opt(B,\succsim)$. Suppose $x \not\in C\opt(A,\succsim)$. Then by Proposition~\ref{prop:rational_choice_nonempty}, there exists $y \in C\opt(A,\succsim)$. Note that $x,y \in A = A \cap B$, $x \in C\opt(B,\succsim)$ and $y \in C\opt(A,\succsim)$. By HWARP, $x \in C\opt(A,\succsim)$, which is a contradiction.
	\end{itemize}
\end{proof}

\begin{proposition}\label{prop:hwarp_alpha_beta_equiv}
	The following are equivalent for $C\opt(\cdot,\succsim)$, where $C\opt: \mathcal{P}(X) \to \mathcal{P}(X)$
	\begin{itemize}
		\item[(i)] $\succsim$ is rational
		\item[(ii)] $C\opt$ satisfies Sen's $\alpha$ and $\beta$
		\item[(iii)] $C\opt$ satisfies HWARP
	\end{itemize}
\end{proposition}
\begin{proof}
	(ii) and (iii) are equivalent by Proposition~\ref{prop:alpha_beta_hwarp}. (i) $\Longrightarrow$ (ii) is given by Propositions~\ref{prop:rational_alpha} and \ref{prop:rational_beta}. Finally, (iii) $\Longrightarrow$ (i) is given below, in the proof of Proposition~\ref{prop:warp_rationality}.
\end{proof}

\subsection{Observed Choice}

Recall the formal definition of choice correspondences above. We will now add some more structure to that definition.

\begin{definition}
	For $\mathcal{B}$ a collection of subsets of $X$, $(\mathcal{B},C)$ is called a \blue{choice structure} if $C(B) \subseteq B$ and $C(B) = \emptyset \Longleftrightarrow B = \emptyset$ for all $B \in \mathcal{B}$.
\end{definition}

\begin{definition}
	The choice structure $(\mathcal{B},C)$ satisfies the \blue{weak axiom of revealed preference} (\blue{WARP}) if for all $A,B \in \mathcal{B}$ where $x$ and $y$ are in both $A$ and $B$, $x \in C(A)$, and $y\in C(B)$, then $x \in C(B)$ and $y \in C(A)$.\footnote{Note the difference in wording from before -- we cannot have as a condition that $x,y \in A \cap B$ as $A \cap B$ is not necessarily in $\mathcal{B}$.}
\end{definition}
\begin{remark}
	When $\mathcal{B} = \mathcal{P}(X)$, WARP is the same as HWARP.
\end{remark}
\begin{definition}
	Given a choice structure $(\mathcal{B},C)$, the \blue{revealed preference relation} $\succsim\opt$ is defined such that $x \succsim\opt y$ if $\exists B \in \mathcal{B}$ such that $x,y \in B$ and $x \in C(B)$.
\end{definition}
\begin{proposition}\label{prop:warp_rationality}
	Suppose that $X$ is finite and $\mathcal{B} = \mathcal{P}(X)$. If $(\mathcal{B},C)$ satisfies WARP then the revealed preference relation that it induces, $\succsim\opt$ is rational and $C(B) = C\opt(B,\succsim\opt)$ for all $B \in \mathcal{B}$.
\end{proposition}
\begin{proof}
	If $\mathcal{B} = \mathcal{P}(X)$ and $(\mathcal{B},C)$ is a choice structure, then $C(Y)$ is defined as nonempty for every $Y = \{x,y\} \subseteq X$. This implies that $x \succsim\opt y$ or $y \succsim\opt x$ for all $x,y \in X$, so $\succsim\opt$ is complete.
	
	Suppose $x\succsim\opt y$ and $y \succsim\opt z$. Then there exists $A \subseteq X$ containing $x$ and $y$ such that $x \in C(A)$; and $B\subseteq X$ containing $y$ and $z$ such that $y \in C(B)$. Moreover, $\{x,y,z\} \subseteq \mathcal{B}$ and $C(\{x,y,z\})$ is nonempty. Suppose $y \in C(\{x,y,z\})$. Then by WARP, $x \in C(\{x,y,z\})$. Suppose $z \in C(\{x,y,z\})$. Then again by WARP, $y \in C(\{x,y,z\})$ and thus $x \in C(\{x,y,z\})$. In any case, $x \in C(\{x,y,z\})$ implies that $x \succsim\opt z$, so $\succsim\opt$ is transitive.
	
	Let $x$ be an element of $C\opt(B,\succsim\opt)$. Then $x \succsim\opt y \forall y \in B$. Since $C(B)$ is nonempty, we have that $z \in C(B)$ for some $z$. By $x \succsim\opt z$, there exists $A \in \mathcal{B}$ such that $x,z \in A$ and $x \in C(A)$. Therefore by WARP, $x \in C(B)$. Conversely, suppose $x \in C(B)$. Then $x \succsim\opt y$ for all $y \in B$, and so $x \in C\opt(B,\succsim\opt)$.
\end{proof}

\begin{remark}
	A stronger version of Proposition~\ref{prop:warp_rationality} exists, though we do not present the proof here:
\end{remark}
\begin{proposition}\label{prop:warp_rationality_3}
	Suppose that $X$ is finite and for all $Y \subseteq X$ where $|Y| \le 3$, $Y \in \mathcal{B}$. If $(\mathcal{B},C)$ satisfies WARP then the revealed preference relation that it induces, $\succsim\opt$ is rational and $C(B) = C\opt(B,\succsim\opt)$ for all $B \in \mathcal{B}$.
\end{proposition}

\begin{remark}
	This does not hold for anything less strong than 3. Consider the following counterexample: Suppose $X = \{x,y,z,w\}$ and $\mathcal{B} = \{\{x,y\},\{y,z\},\{z,w\},\{w,x\}\}$. Let $C$ be defined by:
	\[
	C(\{x,y\}) = \{x,y\} \quad ;\quad C(\{y,z\}) = \{y,z\} \quad ;\quad C(\{z,w\}) = \{z,w\} \quad ;\quad C(\{w,z\}) = \{x\} 
	\]
	Because no pair of elements of $X$ are both in two elements of $\mathcal{B}$, WARP is vacuously satisfied. But neither $x \succsim\opt z$ or $z \succsim\opt x$, so $\succsim\opt$ is incomplete. We can also show that it is intransitive (how?). Moreover, if we extend $C$ to the family of all two-element subsets of $X$, such that everything except for $\{w,x\}$ is mapped to itself (and $C(\{w,z\}) = \{x\}$), $\succsim\opt$ is complete but remains intransitive.
\end{remark}

\subsection{Incomplete Preferences}

\begin{definition}
	$\succ$ is a \blue{strict partial order} if (i) for any $x,y \in X$, if $x \succ y$, then $y \not\succ x$, and (ii) $\succ$ is transitive.
\end{definition}
\begin{remark}
	Note that we are explicitly not defining $\sim$ as $x \sim y$ if $x \not\succ y$ and $y \not\succ x$. The two elements could be incomparable, we do not assume completeness here.
\end{remark}

\begin{proposition}\label{prop:partial_order}
	Define choice by
	\[
	C\opt(A,\succ) \coloneqq \{x \in A : \forall y \in A, y\not\succ x\}
	\]
	where $\succ$ is a strict partial order. Then $C$ satisfies Sen's $\alpha$ but not Sen's $\beta$.
\end{proposition}
\begin{proof}
	
	\begin{itemize}
		\item[(i)] Suppose $x \in A \subseteq B$ and $x \in C(B,\succ)$. Then there does not exist $y \in B$ such that $y \succ x$. It follows that no such $y$ exists in $A \subseteq B$ either, so $x \in C(A,\succ)$.
		
		\item[(ii)] Suppose that $x,y \in C(A,\succ)$, $A \subseteq B$, $y \in C(B,\succ)$, and there is some $z \succ x$ in $B$ such that $y$ and $z$ are incomparable. Then the hypotheses of Sen's $\beta$ are satisfied, but $x \not\in C(B,\succ)$.
	\end{itemize}
\end{proof}

\subsection{WARP and the Slutsky Matrix}

We will make the following assumptions throughout:

\begin{assumption}
	We have (i) $L$ commodities, $x \coloneqq (x_1,\dots,x_L) \in \reals^L_+$; (ii) prices $p \coloneqq (p_1,\dots,p_L) \in \reals^L_{++}$; (iii) wealth $w > 0$; and (iv) budget set $B_{p,w} \coloneqq \{x \in \reals^L_+ : p\cdot x \le w\}$.
\end{assumption}

\begin{definition}
	We define the \blue{Walrasian demand function} (also sometimes called the \blue{Marshallian demand function}) by $x : \reals^L_{++} \times \reals_{++} \to \reals^L_+$, where $x(p,w)$ is the consumer's choice at prices $p$ and wealth $w$. Note that $(p,w)$ may not uniquely specify a value. In that case, we have the \blue{Walrasian (Marshallian) demand correspondence}, $X: \reals^L_{++} \times \reals_{++} \rightrightarrows \reals^L_+$.
\end{definition}

\begin{assumption}
We will almost always make the following assumptions on $x$:
\begin{itemize}
	\item[(i)] $x(p,w)$ is homogeneous of degree 0, meaning that
	\[
	x(\alpha p ,\alpha w) = x(p,w) \text{ for all } (p,w) \in \reals^L_{++} \times \reals_{++} \text{ and } \alpha > 0
	\]
	\item[(ii)] $x(p,w)$ satisfies Walras' Law: $p \cdot x(p,w) = w$ for all $(p,w) \in \reals^L_{++} \times \reals_{++}$
\end{itemize}
\end{assumption}
\begin{proposition}\label{prop:walras_law_choice_structure}
	Let $\mathcal{B}^W \coloneqq \{B_{p,w} : (p,w) \in \reals^L_{++} \times \reals_{++}\}$ and $C_x(B_{p,w}) \coloneqq \{x(p,w)\}$, and let $x$ be homogeneous of degree 0 and satisfy Walras' Law. Then $(\mathcal{B}^W,C_x)$ is a choice structure.
\end{proposition}
\begin{proof}
	We want to show that $C_x(B_{p,w})$ is a uniquely-defined nonempty subset of $B_{p,w}$ for all $B_{p,w} \in \mathcal{B}^W$. That $C_x(B_{p,w})$ is nonempty follows from the definition of $x$ as a function (or correspondence). Homogeneity of degree 0 implies that for $B_{p,w} = B_{\alpha p,\alpha w}$, $C_x(B_{p,w}) = C_x(B_{\alpha p,\alpha w})$. Walras' Law implies that $C_x(B_{p,w}) \subseteq B_{p,w}$.
\end{proof}

\begin{definition}
	In the context of consumer choice, $x(p,w)$ satisfies the \blue{weak axiom of revealed preferences} (\blue{WARP}) if the following holds: If $(p,w),(p',w') \in \reals^L_{++} \times \reals_{++}$ are such that $p' \cdot x(p,w) \le w'$ and $x(p',w') \ne x(p,w)$, then $p\cdot x(p',w') > w$.
\end{definition}
\begin{remark}
	Basically, if the consumer ever chooses $x'$ when $x$ is available, then there's no way that both $x$ and $x'$ could be available and $x$ would be chosen.
\end{remark}
\begin{definition}
	A \blue{Slutsky compensated price change} is a price change from $p$ to $p'$ accompanied by a change in wealth from $w$ to $w'$ that makes the old bundle just affordable. That is, such that $p' \cdot x(p,w) = w'$.
\end{definition}

\begin{proposition}
	\red{(Law of Compensated Demand)} Suppose that consumer demand $x(p,w)$ is homogeneous of degree 0 and satisfies Walras' Law. Then $x(p,w)$ satisfies WARP if and only if for any compensated price change from $(p,w)$ to $(p',w') \coloneqq (p',p' \cdot x(p,w))$ we have
	\[
	(p' - p)\cdot (x(p',w') - x(p,w)) \le 0
	\] 
	with strict inequality if $x(p',w') \ne x(p,w)$.
\end{proposition}
\begin{proof}
	By WARP, $p \cdot x(p',w') \ge p \cdot x(p,w) = w$, with strict inequality if and only if $x(p,w) \ne x(p',w')$. By Walras' Law, we have that $p'\cdot x(p',w') = p'\cdot x(p,w) = w'$. Subtracting, we get
	\[
	(p - p') \cdot x(p',w') \ge (p - p')\cdot x(p,w) \Longrightarrow (p' - p) \cdot (x(p',w') - x(p,w)) \le 0
	\]
	Conversely, say that $(p' - p) \cdot (x(p',w') - x(p,w)) \le 0$. Then we have that
	\[
	p' \cdot x(p',w') - p' \cdot x(p,w) - p \cdot (x(p',w') - x(p,w)) \le 0 \Longrightarrow p \cdot x(p',w') > w
	\]
	since $p' \cdot x(p',w') < p' \cdot x(p,w)$. The case of strict inequality is analogous.
\end{proof}

\begin{proposition}\label{prop:law_of_demand}
	Let $x : \reals^L_+ \times \reals_+ \to \reals^L_+$ be continuously differentiable. Then
	\[
	\frac{\partial x_j(p,w)}{\partial p_j} + x_j(p,w) \frac{\partial x_j(p,w)}{\partial w} \le 0
	\]
\end{proposition}
\begin{proof}
	Assume that $p$ changes solely in $p_j$, by $\Delta p_j > 0$, and let $\Delta w$ be the compensating change in wealth, as above. Let $\Delta x \coloneqq x(p',w') - x(p,w)$. Then by the Law of Compensated Demand, we have that
	\[
	\Delta p_j (x_j(p',w') - x_j(p,w)) \le 0 \Longrightarrow \frac{x_j(p',w') - x_j(p,w)}{\Delta p_j} \le 0
	\]
	Adding and subtracting $x_j(p',w)$, this becomes
	\[
	\frac{x_j(p',w) - x_j(p,w)}{\delta p_j} + \frac{x_j(p',w') - x_j(p',w)}{\Delta p_j} \le 0
	\]
	Using the fact that $\Delta w = \Delta p_j x_j(p,w)$, we get that
	\[
	\frac{x_j(p',w) - x_j(p,w)}{\delta p_j} + x_j(p,w)\frac{x_j(p',w') - x_j(p',w)}{\Delta w} \le 0
	\]
	Taking the limit as $\Delta p_j \searrow 0$, which implies that $\Delta w \searrow 0$ and $p' \to p$), and using the fact that $x$ is continuously differentiable, this becomes
	\[
	\frac{\partial x_j(p,w)}{\partial p_j} + x_j(p,w) \frac{\partial x_j(p,w)}{\partial w} \le 0
	\]
\end{proof}

\begin{definition}
	The \blue{Slutsky matrix} is the matrix of the partials defined above:
	\begin{align*}
		S(p,w) &\coloneqq D_p x(p,w) + D_w x(p,w) x(p,w)^T \\
		&= \matrixc{\frac{\partial x_1}{\partial p_1} + x_1 \frac{\partial x_1}{\partial w} & \cdots & \frac{\partial x_1}{\partial p_L} + x_L \frac{\partial x_1}{\partial w} \\ \vdots && \vdots \\\frac{\partial x_L}{\partial p_1} + x_1 \frac{\partial x_L}{\partial w} & \cdots & \frac{\partial x_L}{\partial p_L} + x_L \frac{\partial x_L}{\partial w} }
	\end{align*}
\end{definition}

\begin{proposition}\label{prop:slutsky_nsd}
	$S(p,w)$ is negative semi-definite.
\end{proposition}
\begin{proof}
	Let $dp \coloneqq (dp_1,\dots,dp_L)$ be an arbitrary element of $\reals^L$. Then for all $i$, we have that
	\begin{align*}
		dx_i &= \frac{\partial x_i}{\partial p_1}dp_1 + \cdots + \frac{\partial x_i}{\partial p_L}dp_L + \frac{\partial x_i}{\partial w}x_1(p,w)dp_1 + \cdots + \frac{\partial x_i}{\partial w}x_L(p,w)dp_L \\
		\Longrightarrow dx &= (D_px(p,w) + D_wx(p,w) x(p,w)^T) dp
	\end{align*}
	By WARP, $dp \cdot dx \le 0$, meaning that
	\[
	dp^T (D_px(p,w) + D_wx(p,w) x(p,w)^T) dp \le 0
	\]
	Thus, $S(p,w)$ is negative semi-definite, since $dp$ is arbitrary.
\end{proof}


\subsection{Consumer Choice from $\succsim$}

\begin{assumption}
	As before, let $X \coloneqq \reals^L_+$. 
\end{assumption}
\begin{definition}
	A \blue{utility function} representing $\succsim$ on $X$ is a function $u: X \to \reals$ such that for all $x,y \in X$:
	\[
	x \succsim y \Longleftrightarrow u(x) \ge u(y)
	\]
\end{definition}

\begin{proposition}
	If $u: X \to \reals$ represents $\succsim$ on $X$ and $f: \reals \to \reals$ is strictly increasing, then $f \circ u$ represents $\succsim$.
\end{proposition}
\begin{proof}
	\[
	x \succsim y \Longleftrightarrow u(x) \ge u(y) \Longleftrightarrow (f\circ u)(x) \ge (f\circ u)(y)
	\]
\end{proof}

\begin{remark}
	Lexicographic preferences, defined on $\reals^2$ by
	\[
	(x_1,x_2) \succsim (y_1,y_2) \Longleftrightarrow x_1 > y_1 \text{ or } x_1 = y_1 \text{ and } x_2 \ge y_2
	\]
	are rational but cannot be represented by a utility function. Why is that?
\end{remark}

\begin{definition}
	The following mathematical concepts will be useful to us:
	\begin{itemize}
		\item[(i)] The \blue{upper contour set}, $R(x) \coloneqq \{y \in X : y \succsim x\}$, is the set of all bundles that are at least as good as $x$. Denote its complement by $P^{-1}(x)$.
		
		\item[(ii)] The \blue{lower contour set}, $R^{-1}(x) \coloneqq \{y \in X : x \succsim y\}$, is the set of all bundles that $x$ is at least as good as. Denote its complement by $P(x)$.
	\end{itemize}
\end{definition}

\begin{definition}
	The preference relation $\succsim$ on $X$ is \blue{continuous} if $R(x)$ and $R^{-1}(x)$ are closed subsets of $X$ for all $x \in X$.
\end{definition}

\begin{remark}
	Lexicographic preferences are not continuous. Can you show why?
\end{remark}
\begin{proposition}\label{prop:debreu}
	\red{(Debreu's Theorem)} Suppose a preference relation $\succsim$ on $X$ is rational and continuous. Then there is a continuous utility function representing $\succsim$.
\end{proposition}
\begin{proof}
	(Sketch) We will sketch this proof assuming that $\succsim$ satisfy strong monotonicity (defined below), which is not necessary but makes the proof easier. Choose any $x \in X$. By strong monotonicity, $x \succsim 0$. Let $e = (1,1)$. By strong monotonicity, $\exists \alpha \in \reals_+$ such that $\alpha e \succ x$. By strong monotonicity, $\exists \alpha : X \to \reals_+$ such that $\alpha(x)e\sim x \forall x \in X$. 
	
	We claim that $\alpha(\cdot)$ represents $\succsim$. First, suppose that $\alpha(x) \ge \alpha(y)$. Then $\alpha(x)e \succsim \alpha(y)e$ by strict monotonicity, and by transitivity we have that $x \sim \alpha(x)e \succsim \alpha(y)e \sim y \Longrightarrow x \succsim y$. Conversely, assume that $x\succsim y$. Then $\alpha(x)e \sim x \succsim y \sim \alpha(y)e$, so $\alpha(x)e\succsim \alpha(y)e$ by transitivity, and $\alpha(x) \ge \alpha(y)$ by strict monotonicity.
\end{proof}

\begin{definition}
	The preference relation $\succsim$ is \blue{monotone} if for all $x,y \in X$, $x \ge y \Longrightarrow x\succsim y$. It is \blue{strictly monotone} if $x \ge y$ and $x \ne y$ implies that $x \succ y$. Note that the latter implies the former.
\end{definition}

\begin{definition}
	The preference relation $\succsim$ is \blue{locally non-satiated} if for every $x \in X$ and for every $\varepsilon > 0$, there exists $y \in X$ such that $\|x - y\| \le \varepsilon$ and $y \succ x$. Note that strict monotonicity implies local non-satiation.
\end{definition}
\begin{remark}
	We assumed earlier that $X = \reals^L_+$. This concept can be extended to any metric space, replacing the norm with the space's distance function.  
\end{remark}
\begin{definition}
	The preference relation $\succsim$ on $X$ is \blue{convex} if for all $x,y,z\in X$ and all $\alpha \in [0,1]$, $y \succsim x$ and $z \succsim x$ implies that $\alpha y + (1-\alpha)z \succsim x$.
	
	It is \blue{strictly convex} if for all $x,y,z \in X$ and all $\alpha \in (0,1)$, $y \ne z$, $y \succsim x$, and $z \succsim x$ imply that $\alpha y + (1-\alpha)z \succ x$.
\end{definition}

\begin{remark}
	Preferences are convex if and only if $R(x)$ is convex for every $x \in X$. Can you prove this?
\end{remark}

\begin{definition}
	The function $u: X \to \reals$ is \blue{quasiconcave} if for all $x,y \in X$ and any $\alpha \in [0,1]$,
	\[
	u(\alpha x + (1-\alpha)y) \ge \min\{u(x),u(y)\}
	\]
	The function $u: X \to \reals$ is \blue{concave} if for all $x,y \in X$ and any $\alpha \in [0,1]$,
	\[
	u(\alpha x + (1-\alpha)y) \ge \alpha u(x) + (1-\alpha)u(y)
	\]
	Strict quasiconcavity and strict concavity are defined analogously, restricting $\alpha$ to $(0,1)$, requiring that $x \ne y$, and replacing weak inequalities with strict ones.
\end{definition}
\begin{proposition}\label{prop:quasiconcave_convex}
	$u$ representing $\succsim$ is quasiconcave if and only if $\succsim$ is convex.
\end{proposition}
\begin{proof}
	Assuming quasiconcavity, $y,z \succsim z \Longrightarrow u(y),u(z) \ge u(x)$ implies that $u(\alpha y + (1-\alpha)z) \ge \min\{u(y),u(z)\} \ge u(x)$. Conversely, suppose WLOG that $y \succsim z$. Note also that $z \succsim z$. Thus by convexity of preferences, $\alpha y + (1-\alpha)z \succsim z$, meaning that $u(\alpha y + (1-\alpha)z) \ge u(z) = \min\{u(y),u(z)\}$.
\end{proof}

\subsection{Consumer Optimization}

\begin{definition}
	The \blue{consumer's problem} is the optimization problem
	\[
	\max_{x \in \reals^L_+} u(x) \st p \cdot x \le w
	\]
\end{definition}

\begin{proposition}
	\red{(Properties of Walrasian Demand Correspondence)} Let $u$ be a continuous utility function representing $\succsim$ on $\reals^L_+$.
	\begin{itemize}
		\item[(i)] If $p \in \reals^L_{++}$ and $w \in \reals_{++}$, then there exists an $x\opt \in \reals^L_{++}$ that solves the consumer's problem
		\item[(ii)] If $\lambda > 0$, then $x\opt$ also solves the consumer's problem for $\lambda p$ and $\lambda w$ (homogeneity of degree 0)
		\item[(iii)] If in addition $\succsim$ is locally non-satiated, then Walras' Law holds, meaning that $p \cdot x\opt = w$
		\item[(iv)] If in addition $\succsim$ is strictly convex (equiv. $u$ strictly concave) then $x\opt$ is unique and the Walrasian demand function $x : \reals^L_{++} \times \reals_{++} \to \reals^L_+$ is well-defined and continuous.
	\end{itemize}
\end{proposition}
\begin{proof}
	\begin{itemize}
		\item[(i)] $B_{p,w}$ is nonempty and compact and $u$ is continuous, so conclusion follows from the Extreme Value Theorem.
		
		\item[(ii)] Observe that $p \cdot x \le w \Longleftrightarrow \lambda p \cdot x \le \lambda w$, so the constraint set is the same in both problems.
		
		\item[(iii)] Suppose not: $p \cdot x\opt < w$. Choose $\varepsilon > 0$ such that $p \cdot y < w$ for all $y \in B_\varepsilon(x\opt)$. By local non-satiation, there exists $y \in B_\varepsilon(x\opt)$ such that $y \succ x\opt$. This is a contradiction.
		
		\item[(iv)] Suppose not: let $\hat{x}$ be a distinct solution. Fix $\alpha \in (0,1)$. By strict convexity of preferences, $\alpha \hat{x} + (1-\alpha)x\opt \succ x\opt$. By convexity of the budget set, $\alpha \hat{x} + (1-\alpha)x\opt$ is affordable, contradicting that $x\opt$ is a global maximum. Continuity of $x$ is annoying but proven elsewhere.
	\end{itemize}
\end{proof}

\begin{proposition}\label{prop:necessary_mrs}
	\red{(Necessary Conditions)} Suppose that
	\begin{itemize}
		\item[(i)] The consumer's preferences on $\reals^L_+$ can be represented by a twice continuously differentiable utility function $u$.
		
		\item[(ii)] The preferences are strictly monotone.
		
		\item[(iii)] $p \gg 0$ and $w \gg 0$.
	\end{itemize}
	If $x\opt$ is an interior solution to the consumer's problem (\ie $x\opt \gg 0$), then
	\[
	\text{MRS}_{ij}(x\opt) \coloneqq \frac{\frac{\partial u(x\opt)}{\partial x_i}}{\frac{\partial u(x\opt)}{\partial x_j}} = \frac{p_i}{p_j}
	\]
\end{proposition}
\begin{proof}
	Strict monotonicity implies that $p \cdot x\opt = w$ and $\frac{\partial u(x\opt)}{\partial x_j} > 0$. We know that $x\opt$ solves the consumer's problem, and the constraint qualification holds. By the Karush-Kuhn-Tucker Theorem, there exists $\lambda > 0$ such that $\nabla u(x\opt) = \lambda p$. Conclusion follows.
\end{proof}

\begin{proposition}\label{prop:sufficient_unique}
	\red{(Sufficient Conditions)} Suppose in addition to hypotheses (i) to (iii) of Proposition~\ref{prop:necessary_mrs}, we have 
	\begin{itemize}
		\item[(iv)] $\succsim$ are strictly convex. 
	\end{itemize}
	If $x\opt$ satisfies $x\opt \gg 0$ and $p \cdot x\opt = w$, and there exists $\lambda > 0$ such that $\nabla u(x\opt) = \lambda p$, then $x\opt$ is the unique solution to the consumer's problem.
\end{proposition}
\begin{proof}
	Omitted, but covered in detail in Part 6: Static Optimization of Tak's lecture notes.
\end{proof}

\paragraph{Some Math Remarks.} These last few sections make a number of extremely strong assumptions on the shape and size of $X$. These assumptions are largely not necessary, and can trivially be relaxed as far as assuming that $X$ is a metric space. They can be relaxed significantly further than that, with difficulty. If you are interested in what that entails, I can happily talk for hours about it. If you're not a masochist, you can ignore this entire note and assume we are in non-negative Euclidean space always. - Gabe

\newpage
\section{Consumer Theory (Kircher)}\label{sec:kircher}

\newpage
\section{Producer Theory (Harris)}\label{sec:harris}

\newpage
\section{Uncertainty Theory (Blume)}\label{sec:blume}

\newpage
\section{Uncertainty Applications (Barseghyan)}\label{sec:barseghyan}

\newpage
\section{Information Theory (Battaglini)}\label{sec:battaglini}

\newpage
\section{Exercises and Examples}\label{sec:exercises}

\subsection{Choice (Easley)}

\subsubsection{Easley Homework}

\subsubsection{TA Section Examples}

\subsubsection{Outside Questions}





















\end{document}