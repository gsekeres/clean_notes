\documentclass[12pt]{article}

%% Packages
\usepackage[margin=1in, top=0.75in]{geometry}
\usepackage[utf8]{inputenc}
\usepackage[T1]{fontenc}
\usepackage[usenames,dvipsnames]{xcolor}
\usepackage{amssymb, amsfonts, amsmath, mathrsfs, enumitem, tcolorbox, bbm, graphicx, fullpage, parskip, mathtools, float, amsthm}
\usepackage{tikz,sgame,bbm,todonotes, setspace, soul}
\usepackage[english]{babel}

\setcounter{tocdepth}{2}
% Links (and references)
\definecolor{linkblue}{RGB}{40, 50, 200}
\usepackage[colorlinks=true, allcolors={linkblue}]{hyperref}

%% Math operators
\newcommand*{\ones}{\text{\usefont{U}{bbold}{m}{n}1}}
\newcommand{\reals}{\mathbb{R}}
\newcommand{\rationals}{\mathbb{Q}}
\newcommand{\integers}{\mathbb{Z}}
\newcommand{\naturals}{\mathbb{N}}
\newcommand{\complex}{\mathbb{C}}
\newcommand{\normal}{\mathcal{N}}

% General math
\newcommand{\abs}[1]{\mathop{\left|#1\right|}} % absolute value
\newcommand{\inv}{^{-1}} % inverse
\let\oldST\st
\newcommand{\strikethrough}{\oldST}
\renewcommand{\st}{\;\text{s.t.}\;} % math operator for "such that"
\newcommand{\eg}{\emph{e.g.} }
\newcommand{\ie}{\emph{i.e.} }
\newcommand{\interior}{\mathop{\rm int}}

% Optimization
\newcommand{\argmax}{\mathop{\rm argmax}}
\newcommand{\argmin}{\mathop{\rm argmin}}
\newcommand{\opt}{^\star}
% Analysis, vector spaces, and topology
\newcommand{\set}[1]{\left\{#1\right\}} % set notation
\newcommand{\seq}[1]{_{#1}^{\infty}} % add sequence notiation to set (or to a summation symbol for series)
\newcommand{\setless}{\mathop{\backslash}} % A \ B notation
\newcommand{\pow}{\mathop{\mathcal{P}}} % power set
\newcommand{\im}{\mathop{\rm im}} % image
\newcommand{\spans}{\mathop{\rm span}} % span
\newcommand{\rank}{\mathop{\rm rank}} % rank
\newcommand{\topo}{\mathop{\mathcal{T}}} % topology
\newcommand{\cont}{\mathop{\bf C}} % continuously differentiable

% Matrices
\newcommand\colvector[1]{\begin{bmatrix}#1\end{bmatrix}}
\newcommand\rowvector[1]{\begin{bmatrix}#1\end{bmatrix}}
\newcommand\matrixc[1]{\begin{bmatrix}#1\end{bmatrix}}
\newcommand\matrixp[1]{\begin{pmatrix}#1\end{pmatrix}}
\newcommand\detmatrix[1]{\begin{vmatrix}#1\end{vmatrix}}
\newcommand\rankmatrix{\begin{bmatrix}I_r & \rvline & \mathbf{0}_1\\\hline \mathbf{0}_2 & \rvline & \mathbf{0}_3 \end{bmatrix}}

% Statistics
\newcommand{\cov}{\mathop{\rm cov}} % covariance
\newcommand{\corr}{\mathop{\rm corr}} % correlation
\newcommand{\expect}{\mathop{\mathbb{E}}} % expectation
\newcommand{\indep}{\perp \hspace{-1.4ex} \perp} % independence symbol
\newcommand{\distiid}{\mathop{\overset{\text{i.i.d.}}\sim}} % i.i.d.
\newcommand{\oversim}[1]{\mathop{\overset{\text{#1}}\sim}} % general text over \sim
\newcommand{\prob}{\mathbb{P}}
\newcommand{\mse}{\mathop{\rm MSE}}
\newcommand{\var}{\mathop{\rm Var}}
\newcommand{\sd}{\mathop{\rm sd}}
\newcommand{\se}{\mathop{\rm se}}
\newcommand{\bias}{\mathop{\rm bias}}
\newcommand{\toprob}{\overset{p}{\to}}
\newcommand{\toas}{\overset{a.s.}{\to}}
\newcommand{\todist}{\overset{d}{\to}}
\newcommand{\hyp}{\mathbb{H}}

% Economics
\newcommand{\choice}{\mathop{C_{\succsim}}} % choice correspondence

% Update existing operators
\let\oldExists\exists
\renewcommand{\exists}{\oldExists\;}
\let\oldForall\forall
\renewcommand{\forall}{\;\oldForall\;}
\let\oldEmptyset\emptyset
\renewcommand{\emptyset}{\mathop{\varnothing}}
\newcommand{\parl}{\left(}
\newcommand{\parr}{\right)}
\newcommand{\midbar}{\middle|}
\newcommand{\barl}{\left[}
\newcommand{\barr}{\right]}
\newcommand{\curll}{\left\{}
\newcommand{\curlr}{\right\}}


%% Presentation environments
% Proofs, counterexamples, and disproofs
\renewcommand\qedsymbol{$\openbox$}
\renewenvironment{proof}{{\raggedright \textit{\textbf{Proof.}}}}{\qed} % Proof
\newenvironment{pf}{\begin{proof}}{\end{proof}} % Proof (shorthand)

\newenvironment{disproof}{{\raggedright \textit{\textbf{Disproof.}}}}{$\qed$} % Disproof
\newenvironment{counterex}{{\raggedright \textit{\textbf{Counterexample.}}}}{} % Counterexample

% Theorem styles
\theoremstyle{plain}
\newtheorem{result}{Result}
\newtheorem{lemma}{Lemma}[section]
\newtheorem{assumption}{Assumption}[section]
\newtheorem{theorem}{Theorem}[section]
\newtheorem{proposition}{Proposition}[section]
\newtheorem{corollary}{Corollary}[section]
\newtheorem{axiom}{Axiom}[section]
\theoremstyle{definition}
\newtheorem*{example}{Example}
\newtheorem*{definition}{Definition}
\newtheorem*{exercise}{Exercise}
\newtheorem*{model}{Model}
\newtheorem*{proposition*}{Proposition}
\newtheorem*{model*}{Model}
\newtheorem*{solution}{Solution}
\newtheorem*{remark}{Remark}
\newtheorem*{question}{Question}
\newtheorem*{answer}{Answer}
\newtheorem*{algorithm}{Algorithm}

\newcommand{\blue}[1]{\textcolor{blue}{\emph{#1}}}
\newcommand{\red}[1]{\textcolor{red}{\emph{#1}}}




\newcommand{\gabe}[1]{\todo[inline,color=green!20!white]{\textbf{GS:} #1}}


%% Header
\makeatletter
\newcommand{\course}[1]{\def\@course{#1}}
\newcommand{\term}[1]{\def\@term{#1}}
\renewcommand{\title}[1]{\def\@entitle{#1}}
\renewcommand{\maketitle}{
    \begin{tcolorbox}[colframe=darkgray]
        \begin{center}
            \textbf{\@course} \\[0.25em]
            {\Large\textit{\@entitle}} \\[0.5em]
            Gabe Sekeres \\[0.5em]
            \@term
        \end{center}
    \end{tcolorbox}
    \vspace{1em}
}
\makeatother


%% Code
\usepackage{listings}
\usepackage{beramono}
\lstdefinelanguage{Julia}%
  {morekeywords={abstract,break,case,catch,const,continue,do,else,elseif,%
      end,export,false,for,function,immutable,import,importall,if,in,%
      macro,module,otherwise,quote,return,switch,true,try,type,typealias,%
      using,while},%
   sensitive=true,%
   alsoother={$},%
   morecomment=[l]\#,%
   morecomment=[n]{\#=}{=\#},%
   morestring=[s]{"}{"},%
   morestring=[m]{'}{'},%
   breaklines=true,%
}[keywords,comments,strings]%

\lstset{%
    language         = Julia,
    basicstyle       = \ttfamily,
    keywordstyle     = \bfseries\color{blue},
    stringstyle      = \color{magenta},
    commentstyle     = \color{ForestGreen},
    showstringspaces = false,
}






\title{Microeconomics I Notes}
\author{Gabe Sekeres}
\course{ECON 6090}
\term{Fall 2024}
\begin{document}
\maketitle

\section*{Introduction}

I am creating this set of unified notes for ECON 6090: Microeconomics I, as taught at Cornell University in the Fall 2024 semester. Due to unforseen departmental circumstances, this course was taught by six different professors (\href{https://easley.economics.cornell.edu/}{David Easley}, \href{https://philippkircher.com/}{Philipp Kircher}, \href{https://adamharris.phd/}{Adam Harris}, \href{https://sites.santafe.edu/~leb/}{Larry Blume}, \href{https://barseghyan.economics.cornell.edu/}{Levon Barseghyan}, and \href{https://www.mbattaglini.com/}{Marco Battaglini}). This structure necessarily created some confusion in notation and material, so these notes function as my attempt to create a universe of the material we learned.

I rely heavily on the notes created from Prof. Easley's course, which were originally compiled by \href{https://julienneves.com/}{Julien Manuel Neves} and subsequently updated by \href{https://ruqing-xu.github.io/}{Ruqing Xu} and \href{https://economics.cornell.edu/patrick-ferguson}{Patrick Ferguson}. I additionally rely on notes and slides provided by Prof. Harris, slides provided by Prof. Blume, slides from \href{https://blogs.cornell.edu/odonoghue/}{Ted O'Donoghue} provided by Prof. Barseghyan, and notes provided by Prof. Battaglini. These notes are supplemented with the canonical \href{https://global.oup.com/academic/product/microeconomic-theory-9780195073409?cc=us&lang=en&}{Microeconomic Theory} textbook by \href{https://www.upf.edu/web/andreu-mas-colell}{Andreu Mas-Colell}, \href{https://mitsloan.mit.edu/faculty/directory/michael-whinston}{Michael Whinston}, and \href{https://www.hbs.edu/faculty/Pages/profile.aspx?facId=6466}{Jerry Green} (hereafter, MWG); my preferred analysis textbook, \href{https://store.doverpublications.com/products/9780486477664}{Foundations of Mathematical Analysis} by \href{https://condor.depaul.edu/~rjohnson/}{Richard Johnsonbaugh} and \href{https://www.mathgenealogy.org/id.php?id=12494}{W.E. Pfaffenberger}; and the excellent Mathematics notes provided by \href{https://www.takumahabu.com/Economics}{Takuma Habu}. All mistakes are my own.

I will occasionally make reference to the Stanford ECON 202 notes, created by \href{https://www.gsb.stanford.edu/faculty-research/faculty/jonathan-levin}{Jonathan Levin}, \href{https://web.stanford.edu/~isegal/}{Ilya Segal}, \href{https://milgrom.people.stanford.edu/}{Paul Milgrom}, and \href{https://sites.google.com/site/ravijagadeesan/}{Ravi Jagadeesan}. This will mainly be if there exists intuition that I believe is helpful.

\paragraph{Notation.} A large part of this project is an attempt to unify the notation used by our separate professors. I default to the notation used in the Easley notes, then to MWG, and then use my own judgement. New definitions will have a word highlighted in \blue{blue}, and theorems will be named in \red{red}. 

\paragraph{Structure.} The course (and these notes) are organized as follows. Prof. Easley taught an introduction to choice theory, Section~\ref{sec:easley}. Prof. Kircher taught consumer theory, Section~\ref{sec:kircher}. Prof. Harris taught producer theory, and some concepts of market failures, Section~\ref{sec:harris}. Prof. Blume introduced the theory of choice under uncertainty, Section~\ref{sec:blume}, and Prof. Barseghyan continued with theoretical applications for uncertainty and expected utility maximization, Section~\ref{sec:barseghyan}. Prof. Battaglini taught on information theory, Section~\ref{sec:battaglini}. We also include here exercises with solutions, divided into the various sections and sources. This is Section~\ref{sec:exercises}.


\newpage
\section{Choice (Easley)}\label{sec:easley}

\subsection{Preference Theory}

\begin{assumption}
	Let $X$ be a finite set of objects. 
\end{assumption}

\begin{definition}
	Define $\succsim$, a \blue{preference relation} on $X$, as $x \succsim y \Longleftrightarrow x$ is \emph{at least as good as} $y$, for $x,y \in X$. $\succsim$ is a binary relation.
\end{definition}

\begin{definition}
	$x$ is \blue{strictly preferred} to $y$, denoted as $x \succ y$, if $x \succsim y$ and $y \not\succsim x$.
\end{definition}
\begin{definition}
	$x$ is \blue{indifferent} to $y$, denoted as $x \sim y$, if $x \succsim y$ and $y \succsim x$.
\end{definition}

\begin{definition}
	A preference relation $\succsim$ is \blue{complete} if $\forall x,y \in X$, either $x\succsim y$, $y \succsim x$, or both.
\end{definition}
\begin{definition}
	A preference relation $\succsim$ is \blue{transitive} if, $\forall x,y,z \in X$ where $x \succsim y$ and $y \succsim z$, $x \succsim z$.
\end{definition}
\begin{definition}
	A preference relation $\succsim$ is \blue{rational} if it is complete and transitive.
\end{definition}
\begin{remark}
	Prof. Easley takes some issues with this definition. The main issue is that there is an English word `rational' that has absolutely nothing to do with it. Hereafter, always read rational as `complete and transitive'.
\end{remark}

\begin{remark}
	These are all of the abstract concepts in choice theory! From here, we will apply them, and see what we can get. 
\end{remark}

\begin{definition}
	(Informal) Define a \blue{choice structure} $C\opt$ over subsets $B \subseteq X$ as $C\opt(B,\succsim) \coloneqq \{x \in B : x \succsim y \forall y \in B\}$.
\end{definition}
\begin{remark}
	Some direct implications:
	\begin{itemize}
		\item[(i)] If $x \in C\opt(B,\succsim)$ and $y \in C\opt(B,\succsim)$, then $x \sim y$.
		
		\item[(ii)] Suppose that $x \in B$, $x \not\in C\opt(B,\succsim)$, and $C\opt(B,\succsim) \ne \emptyset$. Then there exists $y \in B$ such that $y \succ x$. 
	\end{itemize}
\end{remark}

We will now formalize the above.

\begin{definition}
	Let the \blue{power set} of $X$, denoted $\mathcal{P}(X)$, be the set of all subsets of $X$. Note that since $X$ is finite, $\mathcal{P}(X)$ is finite.
\end{definition}

\begin{definition}
	(Formal) A correspondence $C\opt : \mathcal{P}(X) \rightrightarrows X$ is a \blue{choice correspondence} for some (not necessarily complete; not necessarily transitive) preference relation $\succsim$ if $C\opt(B) \subseteq B$ for all $B \subseteq X$.
\end{definition}
\begin{remark}
	This definition is from the Stanford notes -- I find it more intuitive than defining it the other way, but it requires divorcing the choice structure from the preference relation. Some intuition that's helpful for me: Easley's definition starts with the preference relation and then defines the choice correspondence, while Segal's definition starts with the choice correspondence and then applies it to a preference relation. They will (as we will see below) often be equivalent, but it's a subtle distinction. I will denote an arbitrary choice correspondence by $C\opt(\cdot)$ and one connected with a preference relation $\succsim$ by $C\opt(\cdot, \succsim)$.
\end{remark}

\begin{proposition}\label{prop:rational_choice_nonempty}
	If $\succsim$ is a rational preference relation on $X$, then
	\[
	C\opt : \mathcal{P}(X) \setminus \emptyset \to \mathcal{P}(X) \setminus \emptyset 
	\]
	In words, the associated choice correspondence to a rational preference relation is nonempty for nonempty inputs.
\end{proposition}
\begin{remark}
	The Easley notes define power sets slightly differently. This is unnecessary and (I feel) less intuitive.
\end{remark}
\begin{proof}
	Proof by induction on $n = |B|$. Suppose $|B| = 1$, so $B = \{x\}$ for some $x \in X$. Then by completeness, $x \succsim x$, and $C\opt(B,\succsim) = \{x\} \in \mathcal{P}(X) \setminus \emptyset$. Suppose next that for any $Y$ where $|Y| = n$, $C\opt(Y,\succsim)$ is nonempty. Take some arbitrary $B$, where $|B| = n + 1$. Define $B'\coloneqq B \setminus \{x\}$, and let $x'$ be an element of $C\opt(B',\succsim)$, which is nonempty by the inductive hypothesis. By completeness, either $x \succ x'$, $x' \succ x$, or $x \sim x'$. Case by case, we would have that $C\opt(B,\succsim) \in \curll\{x\},C\opt(B',\succsim),C\opt(B',\succsim) \cup \{x\}\curlr \subseteq \mathcal{P}(X)$, by transitivity.
\end{proof}

\begin{definition}
	$C\opt$ satisfies \blue{Sen's $\alpha$}, also known as \blue{independence of irrelevant alternatives}, if $x \in A \subseteq B$ and $x \in C\opt(B,\succsim)$ implies that $x \in C\opt(A,\succsim)$.
\end{definition}
\begin{remark}
	The classical example of a preference relation that violates Sen's $\alpha$ is `choosing the second-cheapest wine.' It should be fairly clear to see why this violates Sen's $\alpha$. Is it a rational preference relation?
\end{remark}

\begin{proposition}\label{prop:rational_alpha}
	If $\succsim$ is a rational preference relation, then $C\opt(\cdot,\succsim)$ satisfies Sen's $\alpha$.
\end{proposition}
\begin{proof}
	The result is trivially true if $A = B$. Suppose that $A \subset B$. Let $x \in C\opt(B,\succsim)$. Then $x \succsim y$ for all $y \in B$. In particular, if $y \in A \subseteq B$, then $x \succsim y$. Thus, $x \in C\opt(A,\succsim)$.
\end{proof}

\begin{definition}
	$C\opt$ satisfies \blue{Sen's $\beta$}, also known as \blue{expansion consistency}, if $x,y \in C\opt(A,\succsim)$, $A \subseteq B$, and $y \in C\opt(B,\succsim)$ implies that $x \in C\opt(B,\succsim)$.
\end{definition}

\begin{remark}
	I couldn't find a classical example violating Sen's $\beta$, but a simple one is as follows: assume that the waiter offers you French or Italian wine. You are indifferent between them, but then they remember that they also have California wine. You say 	`in that case, I'll have the French wine'. Again, this directly violate's Sen's $\beta$, but is it rational? Why or why not?
\end{remark}

\begin{proposition}\label{prop:rational_beta}
	If $\succsim$ is a rational preference relation, then $C\opt(\cdot,\succsim)$ satisfies Sen's $\beta$.
\end{proposition}
\begin{proof}
	Let $x,y \in C\opt(A,\succsim)$, $A \subseteq B$, and $y \in C\opt(B,\succsim)$. Since $x \in C\opt(A,\succsim)$, we have $x \succsim y$ since $y \in A$. Since $y \in C\opt(B,\succsim)$, we have $y\succsim z$ for all $z \in B$. By transitivity, $x \succsim y$ and $y \succsim z$ implies that $x \succsim z$ for all $z \in B$, so $x\in C\opt(B,\succsim)$. 
\end{proof}

\begin{definition}
	$C\opt$ satisfies \blue{Houthaker's weak axiom of revealed preference} (often called either \blue{HWARP} or \blue{HARP}) if for all $A,B \in \mathcal{P}(X)$ if $x,y \in A \cap B$, $x \in C\opt(A,\succsim)$ and $y \in C\opt(B,\succsim)$, then $x \in C\opt(B,\succsim)$ and $y \in C\opt(A,\succsim)$.
\end{definition}

\begin{proposition}\label{prop:alpha_beta_hwarp}
	$C\opt : \mathcal{P} \rightrightarrows X$ satisfies Sen's $\alpha$ and $\beta$ if and only if it satisfies Houthaker's weak axiom of revealed preference.
\end{proposition}
\begin{proof}
	
	\begin{itemize}
		\item[(i)] ($\alpha + \beta \Longrightarrow $ HWARP) Suppose $x,y \in A \cap B \subseteq \mathcal{P}(X)$, $x \in C\opt(A,\succsim)$, and $y \in C\opt(B,\succsim)$. By Sen's $\alpha$, both $x$ and $y$ are in $C\opt(A \cap B,\succsim)$. Then by Sen's $\beta$, $x \in C\opt(B,\succsim)$ and $y \in C\opt(A,\succsim)$.
		
		\item[(ii)] (HWARP $\Longrightarrow \beta$) Say $x,y \in C\opt(A,\succsim)$, $A \subseteq B$ and $y \in C\opt(B,\succsim)$. Because $A = A \cap B$, $x,y \in C\opt(A \cap B,\succsim)$. Applying HWARP, we have that $x \in C\opt(B,\succsim)$.
		
		\item[(iii)] (HWARP $\Longrightarrow \alpha$) Say $x \in A \subseteq B$ and $x \in C\opt(B,\succsim)$. Suppose $x \not\in C\opt(A,\succsim)$. Then by Proposition~\ref{prop:rational_choice_nonempty}, there exists $y \in C\opt(A,\succsim)$. Note that $x,y \in A = A \cap B$, $x \in C\opt(B,\succsim)$ and $y \in C\opt(A,\succsim)$. By HWARP, $x \in C\opt(A,\succsim)$, which is a contradiction.
	\end{itemize}
\end{proof}

\begin{proposition}\label{prop:hwarp_alpha_beta_equiv}
	The following are equivalent for $C\opt(\cdot,\succsim)$, where $C\opt: \mathcal{P}(X) \to \mathcal{P}(X)$
	\begin{itemize}
		\item[(i)] $\succsim$ is rational
		\item[(ii)] $C\opt$ satisfies Sen's $\alpha$ and $\beta$
		\item[(iii)] $C\opt$ satisfies HWARP
	\end{itemize}
\end{proposition}
\begin{proof}
	(ii) and (iii) are equivalent by Proposition~\ref{prop:alpha_beta_hwarp}. (i) $\Longrightarrow$ (ii) is given by Propositions~\ref{prop:rational_alpha} and \ref{prop:rational_beta}. Finally, (iii) $\Longrightarrow$ (i) is given below, in the proof of Proposition~XXX
\end{proof}

\subsection{Observed Choice}



















\newpage
\section{Consumer Theory (Kircher)}\label{sec:kircher}

\newpage
\section{Producer Theory (Harris)}\label{sec:harris}

\newpage
\section{Uncertainty Theory (Blume)}\label{sec:blume}

\newpage
\section{Uncertainty Applications (Barseghyan)}\label{sec:barseghyan}

\newpage
\section{Information Theory (Battaglini)}\label{sec:battaglini}

\newpage
\section{Exercises}\label{sec:exercises}























\end{document}